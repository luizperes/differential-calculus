\documentclass{article}
\usepackage{amsmath}
\usepackage{amssymb}
\usepackage{pgfplots}

\title{1.4 Exponential Functions}
\author{Luiz Peres}
\date{January 16th, 2017}

\begin{document}

\maketitle

1. Consider the graph of $y=e^x$.\\
(a) Find the equation of the graph that results from reflecting about the line $y = 3$.\\\\
\begin{equation}
\begin{split}
    y & = e^x\\
    & \textit{put on zero}\\
    & = e^x - 3 \\
    & \textit{now we reflect the graph}\\
    & = -e^x + 3\\
    & \textit{then, we move it back to its place (3 positions)}\\
    & = -e^x + 6\\
\end{split}
\end{equation}
(b) Find the equation of the graph that results from reflecting about the line $x = 5$.\\\\
\begin{equation}
\begin{split}
    y & = e^x\\
    & \textit{reflect x and put it on zero}\\
    & = e^{x - 5} \\
    & \textit{now we reflect the graph}\\
    & = e^{-x + 5}\\
    & \textit{then, we move it back to its place (5 positions)}\\
    & = e^{-x + 10}\\
\end{split}
\end{equation}
\\\\
2. Find the domain of each function. (Enter your answer using interval notation.)\\
(a)\\
\begin{equation}
\begin{split}
    f(x) & = \frac{64 - e^{x^2}}{1 - e^{64-x^2}}\\
    & D_f\textit{ :  $\mathbb{R}, (1 - e^{64-x^2}) \neq 0$}\\
    & 1 - e^{64-x^2} = 0\\
    & e^{64-x^2} = 1 | (ln)\\
    & ln(e^{64-x^2}) = ln(1)\\
    & 64 - x^2 = 0\\
    & x^2 = -64\\
    & x = \sqrt{64}\\
	&\textit{we must consider -8 and 8, because of the sqrt.}\\
	&\left(-\infty ,\ -8\right)\cup \left(-8,\ 8\right)\cup \left(8,\ \infty \right)    
\end{split}
\end{equation}
\\\\
(b)\\
\begin{equation}
\begin{split}
    f(x) & = \frac{3+x}{e^{cos(x)}}\\
    & \textit{exponential functions are always positive}\\
    & \therefore\\
    & (-\infty, \infty)     
\end{split}
\end{equation}
3. Find the exponential function $f(x) = Cb^x$ whose graph is given.\\
\begin{equation}
\begin{split}
    15 & = Cb^x \\
    C & = \frac{15}{b^1}\\\\
    135 & = Cb^3 \textit{ switch C and the fraction above}\\
    135 & = \frac{15b^3}{b}\\
    135 & = 15b^2\\
    \frac{135}{15} & = b^2\\
    b & = \sqrt{9}\\
    b & = 3\\\\
    15 & = C\times3^1\\
    C & = 3\\\\
    &\textit{The equation is $5(3^x)$}
\end{split}
\end{equation}
4. If $6^x$, show that:\\
$\frac{f(x+h) - f(x)}{h} = 6x(\frac{6^h-1}{h})$\\\\
\begin{equation}
\begin{split}
    & = \frac{f(x+h) - 6^x}{h}\\
    & = \frac{6^{x+h}-6^x}{h}\\
    & = \frac{6^x(6^h)-6^x}{h}\\
    & = \frac{6^x(6^h - 1)}{h}     
\end{split}
\end{equation}
5. A bacteria culture starts with 300 bacteria and doubles in size every half hour.\\
(a) How many bacteria are there after 2 hours?\\
\begin{equation}
\begin{split}
    300 & \times 2^4\\
    300 & \times 16\\
    & 4800
\end{split}
\end{equation}
(b) How many bacteria are there after $t$ hours?
\begin{equation}
\begin{split}
    300 & \times 2^{2t} \textit{ each hour is worth 2.}
\end{split}
\end{equation}
(c) How many bacteria are there after 40 minutes? (Round your answer to the nearest whole number.)\\
\begin{equation}
\begin{split}
    300 & times 2^{\frac{3}{4}}\\
    300 & \sqrt[3]{2^4}\\
    300 & \sqrt[3]{2^3 \times 2}\\
    600 & \sqrt[3]{2}\\
    & \approx 756
\end{split}
\end{equation}
(e) Estimate the time for the population to reach 40,000. (Round your answer to one decimal place.)\\
\begin{equation}
\begin{split}
    300 \times 2^{2t} & = 40000\\
    2^{2t} & = \frac{40000}{300}\\
    2^{2t} & = \frac{400}{3}\\
    \log_2(2^{2t}) & = \log_2(\frac{400}{3})\\
    2t & = \log_2(400) - \log(3)\\
    t & = \frac{8.642356 - 1.584963}{2}\\
    & \approx 3.5 
\end{split}
\end{equation}
\end{document}